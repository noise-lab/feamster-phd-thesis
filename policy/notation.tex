\section{Routing Model and Definitions}\label{sec:notation}
We now define our routing model.  After introducing some basic
terminology, we formally define two notions of good behavior for routing
protocols: {\em stability} and {\em safety}.  Finally, we extend each of
these two definitions to handle filtering expressiveness.

\subsection{Preliminaries}

We consider a model consisting of $N$ ASes (nodes)\footnote{ In this
chapter, we use the terms ``AS'' and ``node'' interchangeably.}, labeled
$1,\ldots,N$.  Each of these nodes wishes to establish a path (defined
below) to a single destination, labeled $0$.

\begin{defn}[Path]
\label{def:path}
A {\em path} from $i$ to $j$ is a sequence of nodes $P = i
i_1 i_2 \ldots i_m j$ with no repetition; \ie, such that $i_u \neq i_v$ if
$u \neq v$, and $i_u \neq i,j$ for all $u$.
%Thus a path is a string of nodes where no node appears more than once.
\end{defn}

Note that we have slightly altered the definition of a {\em path} from
Chapter~\ref{chap:rlogic} (Definition~\ref{defn:rl:path}), although the
concept is conceptually the same: we have eliminated commas and
subscripts for notational convenience, and we have defined that the path
to be loop-free, as BGP typically discards paths that have loops in the
AS-level path.  Definition~\ref{def:path} also precludes the practice of
AS-path prepending (described in Section~\ref{sec:bg:route_selection}),
as it does not affect the results in this chapter.

We denote the number of hops in a path $P$ as $length(P)$.  In addition,
given an AS $k$, we will write $k \in P$ if node $k$ appears in $P$.
For clarity, given a path $P$ from $i$ to $j$, we will often denote $P$
by $iPj$; furthermore, if $P$ is a path from $i$ to $j$, and $Q$ is a
path from $j$ to $k$, then we will denote the concatenation of $P$ and
$Q$ by $iPjQk$.

We denote the set of {\em all} paths from $i$ to $0$ (\ie, all paths
on the complete
graph) using the nodes 
$1,\ldots,N$ by $\P_i^N$.  Given the set of nodes $\{1,\ldots,N\}$, each
AS $i$ will choose a {\em ranking} $\prec_i$ over the set of all paths
$\P_i^N$, defined as follows.

\begin{defn}[Ranking]
A {\em ranking} $\prec_i$ for node $i$ is a total ordering
over the set of all paths $\P_i^N$; thus, given any
two paths $P,Q \in \P_i^N$, either $P \prec_i Q$ ($i$ prefers $Q$ to
$P$) or $P \succ_i Q$ ($i$ prefers $P$ to $Q$).  
\end{defn}

An AS may always choose the {\em empty
path}, $\eps$, which is equivalent to total disconnection from the
destination node $0$.  Thus, we have $\eps \in \P_i^N$ for all $i$ and
$N$.  Furthermore, we assume that every AS strictly prefers
connectivity to disconnectivity, so that $P \succ_i \eps$ for all $P
\in \P_i^N$.  

All paths may not be available to node $i$, due to both
topological constraints and filtering by other nodes.  We will use $\F_i
\subseteq \P_i^N$ to denote the set of paths 
actually available for use by node $i$.  The
empty path is always available; \ie, $\eps \in \F_i$.

A {\em routing system} is specified by the rankings of the individual
nodes, together with the paths available to the individual nodes.
Observe that we have decoupled the ``routing policy'' of
each AS $i$ into two components: the rankings $\prec_i$ of AS $i$
over route advertisements received; and a
determination of which paths are filtered from other ASes.  The
filtering decisions of all nodes, together with 
physical constraints on the network, yield the sets $\F_1, \ldots,
\F_N$.  
We thus have the following formal definition of a routing system.

\begin{defn}[Routing system]
\label{def:system}
A {\em routing system} is a tuple $(N, \prec_1, \ldots, \prec_N,
\F_1, \ldots, \F_N)$, where node $i$ has ranking
$\prec_i$ over the set $\P_i^N$, and $\F_i$ is the set of paths
available to node $i$.
\end{defn}

A routing system specifies the input to any interdomain routing protocol
we might consider.  Given this input, the protocol should converge
to a ``routing tree'': that is, an assignment of a path to each AS, such
that the routes taken together form a spanning tree rooted at $0$.  To
formalize this notion, we must define path assignments and consistent
paths.

\begin{defn}[Path assignment]
A {\em path assignment} for the routing system $(N, \prec_1, \ldots,
\prec_N,\lb 
\F_1, \ldots, \F_N)$ is a vector of paths $\v{P} = (P_1, \ldots, P_N)$ such
that, for all $i$, $P_i \in \F_i$.
\end{defn}

Thus, a path assignment is an assignment of a feasible path to each node
$i$, where feasibility is determined by the set of paths $\F_i$.
Even though each node has a path assigned, these paths may
not be {\em consistent}: node $i$ may be assigned a path
$P_i = ij\hat{P}_j 0$, where $j$ is the first node traversed on $P_i$,
and where $\hat{P}_j$ is a path from $j$ to $0$.  However, the path
$\hat{P}_j$ may not be the same as the path $P_j$ assigned to $j$ in
the path assignment $\v{P}$; in fact, $\hat{P}_j$ may not even be in
the set of feasible paths $\F_j$.  
For example, a node or link along the path $\hat{P}_j$ may experience a
failure, causing the routing protocol to withdraw the path; if $j$ has
heard such a withdrawal but $i$ has not, then it is possible that $P_i =
ij\hat{P}_j 0$ until node $i$ learns that  $\hat{P}_j$ no longer exists.
To formally capture such situations, we define consistent paths and
consistent path assignments.  The definition of consistent path is
a simplification of Definition~\ref{defn:pcp} from
Chapter~\ref{chap:rlogic}, since we do not require the notion of an
induced path in this chapter.

\begin{defn}[Consistent path]
Given a path assignment $\v{P}$, a path $\hat{P}_i$ for node $i$ is
{\em consistent} with $\v{P}$ if one of the following holds:
\begin{enumerate}
\itemsep=-1pt
\item $\hat{P}_i = \epsilon$; or
\item $\hat{P}_i = i0$; or
\item $\hat{P}_i = i j P_j 0$, for some $j \neq i$.
\end{enumerate}
\end{defn}


\begin{defn}[Consistent path assignment]
\label{def:consistent}
A {\em consistent path assignment} for the routing system $(N, \prec_1,
\ldots, \prec_N, \F_1, \ldots, \F_N)$ is a path assignment vector $\v{P}
= (P_1,\ldots,P_N)$ such that for all $i$, $P_i$ is consistent with
$\v{P}$.
\end{defn}


A routing protocol where packets are forwarded solely
on destination should ultimately assign paths that are consistent with
each other.  
%We now formally define a consistent path assignment.

\subsection{Stability and Safety}

Informally, a path assignment is {\em stable} if it is consistent, and no
node has a more preferred consistent path available.  

%% Note that this
%% concept makes no mention of the dynamic process by which a stable
%% point is reached; it only suggests that once a stable path
%% assignment is reached, no further route changes should be made by the
%% nodes.  (We will formalize this point in the next section.) 


\begin{defn}[Stable path assignment]
Given a routing system $(N, \prec_1, \ldots, \prec_N,\lb \F_1,
\ldots, \F_N)$, and a consistent path assignment $\v{P}$, we say
that $\v{P}$ is {\em stable} if for all nodes $i$, and all paths
$\hat{P}_i \in \F_i$
that are consistent with $\v{P}$, $\hat{P}_i \prec_i P_i$.
\end{defn} 

\begin{defn}[Stable routing system]
The routing system $(N, \prec_1, \ldots, \prec_N, \F_1,
\ldots, \F_N)$ is {\em stable} if there exists at least one stable
path assignment $\v{P}$.
\end{defn} 

%\subsection{Safety}

The stability of a
routing system does not indicate whether a routing protocol will
converge {\em regardless} of the initial path assignment.  The {\em
  safety} property, which states that a
protocol eventually converges, regardless of the initial path
assignment and ordering of the routing messages, captures this notion.

In defining safety, we will consider a simplified abstraction of 
BGP.  We model the process by which
nodes receive route advertisements from other nodes and subsequently update
their own route decisions.  We will consider a protocol
dynamic where at each time step only a single AS is {\em activated};
when activated, an AS immediately processes all pending incoming route
advertisements, and then makes a route decision.  Formally, this model will
translate into a path assignment sequence where exactly one node (the
``activated'' node) changes its route at any given time step.

A routing system is safe if no oscillation occurs
regardless of the order in which nodes are activated.  
%We start,
%therefore, by defining a fair activation sequence.

\begin{defn}[Fair activation sequence]
The sequence $i_1, i_2, \ldots$ is a {\em fair activation
sequence} if each node $i = 1,\ldots,N$ appears infinitely often in
the sequence.
\end{defn}

This definition of fair activation sequence is similar to that presented
by Gao and Rexford~\cite{Gao2001a}, except that in our
definition we only 
activate one node at a time.  This distinction is minor: we can
interpret the Gao and Rexford dynamics as a model where 
outstanding routing messages may be in flight when a particular node is
activated.

We now define our simplified model of the routing protocol dynamics: that
is, starting from an initial path assignment $\v{P}_0$, and given a
fair activation sequence of nodes $i_1, i_2,\ldots$, what is the
resulting observed sequence of path assignments $\v{P}_1, \v{P}_2,
\ldots$?  To formalize the dynamics of our model, we consider an
abstraction of the 
BGP decision process described in Figure \ref{fig:bgpdec}.  At each
time $t$, a node $i_t$ is activated, and chooses its most preferred available
path consistent with the path assignment $\v{P}_{t-1}$.  All other
nodes' paths remain unchanged.  It is clear that this decision process
yields a sequence of path assignments $\v{P}_1, \v{P}_2,
\ldots$.

\begin{figure}
\framebox{
\parbox{\linewidth}{
\textit{Routing protocol dynamics}

At time $t-1$, the current path assignment is
$\v{P}_{t-1}$; \ie, each node $i$ has currently selected path
$P_{i,t-1}$ to the destination $0$.  At time $t$:
\begin{enumerate}
\item A given node $i_t$ is activated.
\item Node $i_t$ updates its path to be the {\em most preferred path}
(according to $\prec_{i_t}$) consistent with $\v{P}_{t-1}$.  That
is,
\begin{enumerate}
\item $P_{i_t,t} \in \F_{i_t}\ \textrm{is consistent with}\
  \v{P}_{t-1}$, and
\item $P_{i_t,t} \succ_{i_t}
\hat{P}_{i_t}\ \forall\ \hat{P}_{i_t} \in \F_{i_t}\ \textrm{consistent
with}\ \v{P}_{t-1}$.
\end{enumerate}
\item All other nodes leave their paths unchanged.
% \ie, for $i \neq i_t$,
%\[ P_{i,t} = P_{i,t-1}. \]
\end{enumerate}
}
}
\caption[Model of the routing protocol dynamics]{The routing protocol
  dynamics, given an activation sequence $i_1, 
i_2, \ldots$.  The process starts from an initial path assignment $\v{P}_0$.}
\label{fig:bgpdec}
\end{figure}


After any given activation step $t$, the overall path
assignment $\v{P}_t$ may not be consistent.  Inconsistencies reflect the
fact that a node only updates its path assignment in response to the
receipt of a route advertisement.  If, at time $t_0$, a node $i$ is
using a path that traverses some other node $j$ that has since changed
paths, then node $i$ would obliviously continue to use (and advertise)
that {\em inconsistent} path until it receives a routing update that
reflects that the path through $j$ has disappeared or changed.  
%% When $i$
%% receives the routing update at some later time $t_1$, it selects a new
%% path to the destination from the remaining paths that it knows about at
%% time $t_1$.  Our model uses an activation sequence to reflect $i$'s
%% receipt of routing updates consistent with path assignment $\v{P}_t$ and
%% updating its path to the destination: 
When activated, say, at time $t > t_0$, node $i$ would discover that the
path it was using was 
inconsistent with $\v{P}_t$ and would then select its
highest-ranked path that was consistent with $\v{P}_t$.  The activation
of a node at some time $t$ corresponds to that node receiving all
available routing information in the system up to that time.

With the definition of our protocol dynamics in hand, we can define
protocol {\em safety}.  Given a routing system and an activation
sequence, we say that the system has converged if, after some finite
time, the path assignment remains invariant for all future time.  A
protocol is {\em safe} if it converges to a stable path assignment,
regardless of the initial path assignment and fair activation sequence.

\begin{defn}[Safety]
\label{def:safety}
A routing system $(N, \prec_1, \ldots, \prec_N, \F_1,
\ldots, \F_N)$ is {\em safe} if for any initial path
assignment $\v{P}_0$ and fair activation sequence $i_1, i_2, \ldots$,
there exists a finite $T$ such that $\v{P}_t = \v{P}_T$ for all $t \geq
T$.  
\end{defn} 
 
Because the activation sequences are fair in the preceding
definition, if a routing system converges to $\v{P}_t$, then the
resulting path assignment to which the system converges must be both
consistent and stable.  If not, at least one node would change its path
assignment eventually.  
%Conversely, if the initial path assignment is
%stable, then it is clear that the system will be convergent regardless
%of the activation sequence, since no node will ever alter its choice of
%path.

%Our main interest is in studying routing systems that are always
%convergent, regardless of the activation sequence and initial path
%assignment chosen.  Such systems are called {\em safe}~\cite{Griffin2002c}.
%
%\begin{defn}[Safety]
%\label{def:safety}
%The routing system $(N, \prec_1, \ldots, \prec_N, \F_1,
%\ldots, \F_N)$ is {\em safe} if for any fair activation sequence $i_1,
%i_2, \ldots$ and initial path assignment $\v{P}_0$, the routing system is
%convergent.
%\end{defn}
%
%% Safety is a much more practically interesting condition than stability
%% of a system, since a safe system can be expected to converge even as
%% nodes dynamically update routes.  Since a safe system must have at
%% least one stable path assignment, any safe routing system is also a
%% stable routing system.

\subsection{Filtering}

We are interested in the stability and safety of systems
that result when nodes are allowed to filter routes from other nodes.
We thus require conditions stronger than stability and safety, known as
{\em stability under filtering} and {\em safety under filtering}.
Informally, a routing system is stable (respectively, safe) under
filtering if, under any choices of filters made by the ASes, the
resulting routing system is always stable (respectively, safe).  
%We
%formalize these notions as follows.

\begin{defn}[Stable under filtering]
The routing system $(N, \prec_1, \ldots, \prec_N, \F_1, \ldots, \F_N)$
is {\em stable under filtering} if, for all choices of available paths
$\hat{\F}_i \subseteq \F_i$ for $i = 1,\ldots,N$, the routing system
$(N, \prec_1, \ldots, \prec_N, \hat{\F}_1, \ldots, \hat{\F}_N)$ is
stable.
\end{defn} 

\begin{defn}[Safe under filtering]
The routing system $(N, \prec_1, \ldots, \prec_N, \F_1, \ldots, \F_N)$
is {\em safe under filtering} if, for all choices of available paths
$\hat{\F}_i \subseteq \F_i$ for $i = 1,\ldots,N$, the routing system
$(N, \prec_1, \ldots, \prec_N, \hat{\F}_1, \ldots, \hat{\F}_N)$ is
safe.
\end{defn} 

We interpret these definitions as follows.  The set of available paths
$\F_i$ gives the set of paths that are physically possible for node $i$ to
use, given the current network topology.  Once all nodes have chosen
their route filters, 
$\hat{\F}_i$ gives the set of paths that can ever be used by node $i$ in
route advertisements.  Because we allow arbitrary choice of filters, the
resulting routing system should be stable and safe regardless of the
choices of $\hat{\F}_1, \ldots, \hat{\F}_N$ that are made.



