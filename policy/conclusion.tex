\section{Implications: Possibilities for Guaranteeing
  Safety}\label{sec:policy:implications} 

In light of the results in this chapter, a natural question to ask is
whether they are 
positive or negative.  In one sense, our results are grim, because they
suggest that if BGP remains in its current form and each AS retains
complete autonomy and filtering expressiveness (\ie, the only realistic
scenario for the foreseeable future), then the routing system cannot be
guaranteed to be safe unless each AS constrains its rankings over
available paths to those that are consistent with shortest hop count
(or, alternatively, preferences that are based on consistent edge
weights).  

On the other hand, our results suggest several clear directions for
designing a policy-based routing protocol that is guaranteed to be safe.
Although we presented the ARC function as a proof technique, such a
function could be implemented in practice to restrict the rankings that
operators specify in operational networks.  We foresee a few
possibilities: (1)~the routing protocol remains as it is today, and the
constraints derived in Theorems~\ref{th:unsafe} and~\ref{th:dw} are
checked by a tool that statically detects configuration faults (\eg,
{\em rcc}, as described in Chapter~\ref{chap:rcc}); (2)~the routing protocol
is modified to 
prevent operators from expressing rankings that violate these
constraints in the first place; (3)~ASes use dynamic analysis to detect
when a safety violation has taken place (\eg,~\cite{Griffin2000}); or
(4)~a protocol is designed whereby ASes exchange enough information
about their rankings to enable detection or prevention of safety
violations, but not enough to reveal their strategic business decisions.
We have 
not yet fully evaluated the merits of each approach in this work, but we
briefly speculate on their advantages and disadvantages.

%On the other hand, our results are
%positive, because they suggest a clear direction forward: BGP {\em
%  must} be modified if ASes 
%are to filter without restriction and retain ranking independence,
%without compromising routing safety.  

\subsection{Static Fault Detection with \rcc}

Enforcing the ranking constraints from Section~\ref{ssec:localresults}
locally at each AS using a static analysis tool like \rcc
(Chapter~\ref{chap:rcc}) would require no changes to the existing
routing protocols and configuration languages.  Unfortunately, the
results in Theorems~\ref{th:unsafe} and~\ref{th:dw} only provide global
guarantees if {\em every} AS abides by the constraints: as a result,
there may be little incentive for one AS to constrain its policies if
other ASes do not abide by the same constraints.  As such, while using \rcc
to verify the necessary conditions for safety under filtering could
guarantee safety without requiring any modifications to the protocol, it
would require ASes to abide by constraints that are not enforceable,
since an AS has no incentive to restrict its filtering in this way if
other ASes do not.

\subsection{A New Routing Protocol that Restricts Expressiveness}

Implementing the ranking constraints by restricting the protocol
``knobs'' requires wholesale changes to both the configuration language
and the routing protocol, but it would provide absolute guarantees for
safety while still preserving the autonomy of each AS.  The difficult
design question involves determining exactly {\em how} the knobs should be
restricted.  HLP~\cite{Subramanian2005}, a recent proposal for replacing
BGP, proposes that these knobs constrain the
policy constraints to conform to those suggested by Gao and
Rexford~\cite{Gao2001a}, but the examples in
Section~\ref{sec:bg_stability} suggest that this prescription is too
restrictive.  Our results suggest another possibility: a
protocol that stipulates that rankings be consistent with weighted
shortest paths but affords each AS some flexibility in actually setting
those edge weights.  A recently developed framework called
``metarouting'' may also be useful in designing a protocol that is safe
under filtering~\cite{Griffin2005}.  We now explore each of these
possibilities, which are not mutually exclusive.

\subsubsection{Option \#1: Implement Rankings with Shortest Paths and
  ``Knob Tweaking''}

Our results in
Section~\ref{ssec:ew} suggest one possible direction: there we show that
routing using preferences derived from edge weights is guaranteed to be
stable.  Suppose each AS ranks paths based on the sum of edge weights to
the destination and adjusts weights on its incident outgoing edges to
neighboring ASes.  Rankings would then be derived from the total path
cost, but an AS might still retain enough flexibility to control the
next-hop AS en route to the destination. Such an approach could ensure
that the protocol is safe on short timescales, while allowing ``policy
disputes'' to occur on longer timescales, out-of-band from the routing
protocol.  Of course, we must explore whether this apparently more
restrictive language could still implement the policies that operators
want to express.  Furthermore, a more restrictive policy language would
guarantee safety, but would likely cause routing to oscillate on a
slower timescale as operators observe the routing protocol converging to
undesirable paths.  We believe that this design decision is
exactly the right one: the routing protocol {\em should} converge on a
fast timescale and accurately reflect network topology, while policy
conflicts should be resolved on slower, ``human'' timescales.

One possible problem with routing protocol whose rankings are derived
from consistent edge weights is its lack of modularity. For example, a
failure or policy change in a remote part of the network may require a
network operator to re-tweak the edge weights on the edges incident to
his AS simply to maintain a desired rankings over paths.  Of course, in
practice, an operator would not find such continual knob-tweaking
acceptable.  These adjustments could be performed automatically (\ie,
the weights required to achieve some ranking could be automatically
computed and reconfigured).  Whether such a solution would introduce too
many routing messages into the global routing system, as well as what the
precise technique should be for adjusting these weights, are areas for
future exploration.

\subsubsection{Option \#2: Use ``Metarouting'' to Design a Strictly
  Monotonic Protocol} 

Metarouting~\cite{Griffin2005} is a recently developed framework based
on Sobrinho's path algebras~\cite{Sobrinho2003} that allows a routing
protocol designer to design a routing protocol using a ``metalanguage''
that specifies an algebra.  Informally, an {\em algebra} is a
specification of the labels that links and paths can assume, the output
of composing these labels with some operator, and a ranking function
that specifies which labels are preferred over others.  Sobrinho's work
observed that any {\em strictly monotonic} algebra (\ie, where the
ranking function prefers a path $P_i = (v_i, \ldots, d)$ over any path
$P P_i$) is guaranteed to be safe~\cite{Sobrinho2003}; metarouting shows
how to compose multiple algebras to obtain a resulting algebra that is
strictly monotonic and provides a language for constructing these
algebras. 

Metarouting is a framework for {\em analyzing} the
safety routing protocols, but it remains to be seen whether it can be
used to {\em design} any expressive policy-based routing protocol.  For
example, because next-hop rankings are not monotonic, a routing protocol
that is based on an algebra whose ranking function uses next-hop
rankings as the primary criterion is not safe.  Additionally, a strictly
monotonic routing protocol whose ranking function is not derived from
consistent edge weights still requires restrictions on filtering.  For
the reasons we discussed earlier in this chapter, it is doubtful that
operators would accept any routing protocol that hard-wires filtering
restrictions into the protocol itself.  Further, metarouting only
explores sufficient conditions for safety; it is unclear whether it
captures all interesting routing protocols that satisfy safety
(particularly since it excludes protocols based on next-hop rankings,
for example). 

\subsection{Restrict Autonomy by Exposing Rankings}


Another possibility for guaranteeing safety is to impose
restrictions on autonomy.  Some previous work detects safety violations
using out-of-band distributed detection mechanisms~\cite{Jaggard2004,
Machiraju2004}.  Recent work notes that safety may still be satisfied if
only a single AS deviates from the ranking and filtering constraints of
Gao and Rexford (as summarized in Table~\ref{tab:business}) and presents
a distributed computation that securely determines the number of ASes
that violate the constraints without revealing exactly which ASes are
actually in violation~\cite{Machiraju2004}.  Jaggard and Ramachandran
also proposed mechanisms for out-of-band distributed detection of a
dispute wheel~\cite{Jaggard2004}.

A reasonable area for future work would be to further explore these
types of approaches.  For one, these distributed detection
algorithms do not offer any recourse for {\em resolving} the conflicting
policies that give rise to the safety violation, which would ultimately
require restrictions on autonomy.  For example, suppose every AS were
permitted to express next-hop rankings.  In this case, what information
must each AS reveal (and to whom) about its rankings so that the
resulting system could detect and resolve safety violations?  In a
similar vein, it may be reasonable to exploit the trust relationships
among groups of ASes (\eg, one AS may trust one of its neighbors, but
not others) to design a protocol that guarantees safety but only
requires partial revelation of rankings to some subset of all ASes.

\subsection{Detect Safety Violations using Dynamic Analysis}

Rather than restricting rankings and filters to prevent safety
violations, the routing protocol could rely on analysis of routing
protocol dynamics to detect when safety has been violated.  Previous
work has focused on such techniques.  For example, Griffin \ea proposed
a ``safe path vector protocol''~\cite{Griffin2000}; this chapter suggests
a modification to BGP where each AS includes an explanation of why it
changed routes (\eg, because the route it changed to was more or less
preferred than the route it was previously using).  The protocol
facilitates detection of safety problems, but, in the process, it
exposes the potentially private ranking functions of each AS and
provides no mechanism for resolving conflicts in rankings.  More
generally, analyzing routing protocol dynamics could help identify
safety violations, even without modifications to the routing protocol
itself, although such identification would likely require analysis of
the dynamics from multiple vantage points in the AS graph.


\section{Summary}
\label{sec:policy:conclusion}

This chapter explored the fundamental tradeoff between the
expressiveness of rankings and routing safety, presuming that each AS
retains complete autonomy and filtering expressiveness, and presented
the first study of the effects of filtering on safety.  We make the
following contributions.

\begin{enumerate}
\itemsep=-1pt
\item We showed that next-hop rankings are not safe; we also observed
that although rankings based on a globally consistent weighting of paths
are safe under filtering, even minor generalizations of the weighting
function compromise safety.
\item We defined a {\em dispute ring} and show that any routing system
that has a dispute ring is not safe under filtering.  Our results are
the first necessary conditions concerning safety.
\item We showed that, providing for complete autonomy and filtering
expressiveness, the class of allowable rankings that guarantee safety is
effectively ranking based on variants of weighted shortest paths.  We
also explored the implications of these findings for the design of
future interdomain routing protocols.
\end{enumerate}

This chapter continued the theme of exploring the dynamic properties of
Internet routing by analyzing static properties of the routing
configuration.  The fact that we can determine {\em anything} about the
Internet routing dynamics by analyzing the static rankings of each AS
independently is rather remarkable.  It is even more noteworthy that we
can guarantee safety, a property of the {\em global} routing system, by
analyzing the static configurations of each AS independently.  Of
course, guaranteeing safety does in fact require placing very strict
constraints on each AS's rankings, which is why we advocated relaxing
these constraints to some degree.  An altogether different approach
would be to try to detect these properties at runtime, by analyzing the
routing messages themselves (\ie, {\em dynamic analysis}).  The next
chapter briefly explores how other techniques, such as dynamic analysis,
can complement static analysis.

%% Previous work has
%% demonstrated that policy-based routing protocols (\eg, the Internet's
%% interdomain routing protocol, BGP) may never converge to a stable
%% path assignment if all parties act autonomously in setting their policies and
%% route filters~\cite{Griffin2002c}.  While others have shown that
%% restricting each AS's autonomy in defining its rankings and filters, as
%% well as restrictions on global topology, can
%% guarantee convergence to a stable path assignment~\cite{Gao2001a}, we
%% have argued that these restrictions, particularly those on filtering, are
%% neither verifiable nor enforceable.

%% This paper argues that, since restrictions on filtering are inherently
%% unenforceable, it is reasonable to explore the restrictions on rankings
%% that are necessary to ensure safety, assuming that ASes can filter routes arbitrarily and
%% that each AS acts autonomously in ranking their preferences over
%% available paths.  In exploring this question, this paper presents
%% several contributions.

%% First,   We have shown that
%% seemingly innocuous classes of rankings, such as those based solely on
%% the immediate neighboring AS, or {\em next-hop rankings}, can be unsafe,
%% and that introducing filtering can prevent the routing system from
%% having any stable path assignment whatsoever.  This result is surprising
%% in light of previous results on next-hop rankings, which show that
%% routing systems with only next-hop rankings always have at least one
%% stable path assignment~\cite{Feigenbaum2004}.  We also demonstrate that routing
%% protocols that deviate slightly from routing on consistent edge weights
%% may not be safe.

%% Second, we extend the formalism of Griffin \ea\cite{Griffin2002c} to
%% relate static properties of the routing system to routing protocol
%% dynamics.  Specifically, we build on the ``dispute wheel'' concept by
%% defining a special case of a dispute wheel called a {\em dispute ring}
%% and show that any routing system that has a dispute ring can be unsafe
%% under unrestricted filtering.

%% Finally, we show that, under unrestricted filtering and ranking
%% independence, guaranteeing safety essentially requires implenting
%% shortest path routing.  In other words, the constraints imposed by the
%% realities 

