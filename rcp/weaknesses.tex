\subsection{Challenges Introduced by RCP}\label{sec:weaknesses}

%The RCP architecture potentially introduces limitations and weaknesses;
Separating routing state from the routers can potentially introduce
robustness, scalability, speed, and consistency problems.  The RCP
architecture must address these challenges to be viable.  In this
subsection, we briefly highlight these issues and sketch
possible solutions to these problems.  We are
addressing these problems in greater detail in our current work,
and we are implementing a prototype of RCP using OSPF and BGP
data from AT\&T's domestic IP backbone~\cite{caesar2004}.

It might seem that moving complexity out of the routers
into RCP creates new problems because of the additional flexibility
in path assignment and because we are adding a component to the routing
system.  However, management systems and verification tools
for BGP configuration already exist today, but they are more complicated
and constrained because they must work around the artifacts of today's
routing system.  Thus, adding RCP to the routing system does not
really constitute ``more functionality''; rather, RCP moves
routing functionality to a part of the system where complexity can be
better managed.

%We briefly highlight some of the potential challenges presented by the RCP
%architecture and our proposed solutions to these problems.  We discuss
%these problems in more detail in an accompanying technical
%report~\cite{caesar2004}. 


 {\bf Robustness.} To avoid introducing a single point of failure, RCP
  should be distributed across multiple {\em RCP servers} (RCSes).
  These servers must maintain a consistent view of the available routes
  to ensure that all routers receive consistent, loop-free paths.  The
  RCSes must employ a protocol that recognizes when an AS becomes
  partitioned and guarantees that each partition receives routing
  information that is consistent within its partition.  We are currently
  studying the types of inconsistencies that can result from various
  combinations of partitions.  Our preliminary results suggest that
  even if a network is partitioned, RCSes in separate partitions cannot
  create a forwarding loop.  This result follows from the fact that
  network partitions are caused by partitions of the IGP (\eg, OSPF)
  topology, and RCSes rely on the IGP to exchange routes with each other
  and with BGP routers.  Thus, a protocol that elects an RCS for
  each partition guarantees correct, loop-free forwarding.


%%   Our preliminary results suggest that,
%%   because network partitions result from a partition in the internal
%%   routing protocol (\eg, OSPF), the types of partitions that can cause
%%   RCP-induced forwarding loops are not generally possible and that a
%%   protocol that allows each RCS to determine which RCS learns the most
%%   candidate routes within its partition (and, thus, which RCS is the
%%   ``master'') suffices to guarantee correct, loop-free
%%   forwarding.
%\footnote{Forwarding loops
%  resulting from inconsistent route advertisements require two RCP
%  servers with different routing choices to assign routes to the same
%  set of routers.  In other words, such inconsistencies require RCP
%  servers in two partitions $A$ and $B$ to share some set of routers,
%  but $(A \cup B - A \cap B) \neq \phi$.  We have shown that, if $(A
%  \cup B - A \cap B) \neq \phi$, then the RCP servers in $A$ and $B$
%  cannot be partitioned.}.


%We believe that existing state machine replication
%  or primary/backup schemes may be applicable.

 {\bf Scalability.} RCP must be able to handle thousands of eBGP
  sessions and hundreds of iBGP sessions, each with thousands of routes.
  Today's high-end desktop machines satisfy the memory and computational
  requirements for RCP.  In our current work, we are exploring ways to
  distribute RCP functionality across many physical machines.  One
  design idea we are currently pursuing involves dividing the RCP into a
  {\em BGP engine}, which is responsible for establishing the (possibly
  large number of) BGP sessions to routers within the AS (and,
  ultimately, across ASes) and whose sole responsibility is state
  management; and an {\em RCP engine}, which receives the routing
  information from the machines running BGP engines and implements the
  logic that we have discussed (\eg, path computation,
  configuration management, maintaining consistency, etc.).  
  

 {\bf Convergence speed.} RCP must compute routes using BGP and
  IGP information for every router in the AS and propagate the results
  of this computation in a timely fashion as BGP and IGP topologies
  change.  Because RCP is an active participant in both the BGP and
  IGP protocols, delays due to message passing should be no worse than
  in today's routing architecture.

 {\bf Transient inconsistencies.}  Transient inconsistencies might occur
  if routers do not receive updates from RCP in a certain order. For
  example, if a router's path to a destination includes routers for
  which RCP has already assigned a new path, transient forwarding
  loops could result.  Although this pathology is likely no worse than the
  transient loops that occur during iBGP convergence today, it deserves
  further attention.  In the future, routers might be modified to
  support a ``commit'' operation to allow for all routers along a path
  to execute an update at the same time.


