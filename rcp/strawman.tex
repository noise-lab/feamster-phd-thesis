%%%
%%% strawman.tex
%%%


\subsection{Architectural Principles for Routing}\label{sec:principles}

%% In this subsection, we describe the challenges of managing BGP within a
%% backbone network. Managing intra-AS routing presents three primary
%% challenges: (1)~the outcome intra-AS routing
%% depends on interactions between multiple routing protocols,
%% (2)~operators have indirect control over BGP's behavior, and (3)~because
%% iBGP's routing decisions are decentralized, common network management
%% and diagnosis tasks become more difficult.  The rest of this subsection
%% surveys these problems in further detail.

%% Despite the aspects of BGP that introduce fundamental problems and
%% stifle innovation, little attention has focused on designing protocol
%% improvements that fix these problems.  This lack of attention is largely
%% due to the fact that today's routing architecture does not support
%% innovation.  In this subsection, we describe three architectural principles
%% that will help facilitate innovation in interdomain routing.  In the
%% process, we describe specific aspects of today's routing system that
%% violate these principles.

In this subsection, we present three architectural principles for
reducing interdomain routing complexity:

\begin{enumerate}
\itemsep=-1pt
% 1
\item The routing architecture must base its routing assignments
on a consistent view of routing state. 
% 2
\item The interfaces between the routing protocols must minimize
unexpected or unwanted interactions.
% 3
\item The interdomain routing mechanisms must
directly support flexible, expressive policies.
\end{enumerate}
Each paragraph in this subsection discusses one of these principles.
For each principle, we present a high-level rationale, followed
by specific examples of how today's interdomain routing architecture
violates the principle.  For each of these examples, we suggest how
adhering to the architectural principle helps solve the problem.

%% In this subsection, we propose several architectural principles for
%% interdomain routing, all of which are rooted in traditional systems
%% design arguments, with the following goals in mind: (1)~to enable
%% innovation by fostering a more open platform for interdomain routing;
%% and (2)~to mitigate complexity of the interdomain routing system in
%% order to make network management easier.  To make the deployment of an
%% extensible routing platform feasible, we must keep both
%% goals in mind; in particular, the latter goal is the ``tipping point''
%% that gives operators an incentive to deploy an architecture based on
%% these principles.

\paragraph{Compute Routes Using Consistent State}

Routing state and logic should be co-located with the system components
that are assigning routes.  The logical participants in an interdomain
routing protocol are the {\em ASes\/}, not the individual routers.  The
interdomain routing architecture should view each AS as a single
participant and base routing decisions on a network-wide view of
available routes and configuration state; the routers, on the other
hand, should forward data traffic without concern about how the routes
are computed.  The current interdomain routing system violates this
architectural principle in the following three ways:
%
%The network ``core'' should maintain as little state as possible and the
%intelligence in networks should exist at endpoints.  
%Although the original Internet routing architecture incorporated the
%spirit of the end-to-end argument, 

%This argument holds
%important lessons for interdomain routing.  
%Routing is a distributed program that assigns IP-level paths to each
%destination.  


%First, {\em placing the ``intelligence'' of the routing protocol in
%the routers themselves prevents experimentation with and rapid
%deployment of new features}.  Many of the features of the interdomain
%routing system have ended up in the hands of router vendors because,
%to date, the best (and, arguably, the only) place to add functionality
%to the network is in the routers themselves.  Unfortunately, because
%most innovation at the routing layer involves compliance with
%proprietary hardware and software, as well as the slow-moving
%standards process, innovations to the routing architecture have been
%incremental and difficult to deploy.  Router vendors often add
%functionality in response to requests for new features (such as
%route-flap damping or BGP communities).  Moving the routing protocol
%logic out of the routers to the extent possible would allow protocol
%designers and network operators to easily experiment with and quickly
%deploy protocol modifications, without being tied to router hardware
%and software development cycles.

{\bf Decomposing the routing configuration state across the routers
unnecessarily complicates policy expression}.  Although
distributing state to achieve scalability and reliability makes sense,
many aspects of configuration are not replicated, but rather {\em
decomposed} across routers.  
%This decomposition
%of policy across routers does not improve scalability or reliability,
%but complicates policy expression and makes configuration
%inconsistencies more likely.  
Configuration state should be logically
centralized because it simplifies policy expression without compromising
scalability or reliability.

\vspace{0.05in}
\noindent
{\em Problem:} Network operators must often implement high-level policies,
such as preventing routes learned from one AS from being advertised to
another.  Implementing this policy currently requires modifying the
configurations of multiple routers: the import policies must ``tag''
eBGP-learned routes appropriately, and the export policies of other
routers must filter routes with this tag when advertising to eBGP
neighbors.  

\vspace{0.05in}
\noindent
{\em Solution:} Defining routing policy on a network-wide basis would
obviate the need for this level of indirection.  A network-wide
configuration management entity could know the origin of all routes
based on the eBGP sessions that advertised them, which would allow
a direct expression of policies based on sessions.
\vspace{0.05in}

%Because complexity makes reasoning extremely
%difficult, operators may be less likely to modify configuration because
%they lack the ability to reason about high-level behavior and fear
%introducing mistakes into the configuration.
%Additionally, distributing configuration can introduce
%unnecessary dependencies that actually make the routing protocol more
%error-prone than it should be.  
%This complexity discourages makes operators reluctant to experiment with
%changes and improvements for fear of introducing unintended side effects
%or ``fixing something that isn't broken''.  
%Configuration state should
%be distributed when doing so enhances scalability or reliability;

{\bf Distributed path selection causes routing decisions at one router
  to depend on the configuration of other routers}.  Subtle
  configuration details
  affect the route that a router selects or whether that router learns a
  route at all.  Computing routes on a network-wide basis using a
  consistent view of routing state can reduce interdomain routing's
  dependencies on these subtle details.
%selection can both explicitly enforce correct operation and facilitate
%modeling of the protocol behavior.
%

\vspace{0.05in}
\noindent
{\em Problem:} 
Omitting a single iBGP
session in a full-mesh configuration can leave a router with no route
for certain destinations, even if the intradomain topology is connected.
Distributed path selection also makes 
predicting the effects of configuration changes on traffic
flow difficult~\cite{Feamster2004}. 

\vspace{0.05in}
\noindent
{\em Solution:} 
An entity that performs path assignment on behalf of all
routers could control path assignment to ensure that every router is
assigned a route for every destination.
%For example, consider
%Figure~\ref{fig:today}, which shows how eBGP, iBGP, and IGP
%interoperate.  A router that learns a route on an iBGP session will not
%readvertise that route to routers on other iBGP sessions because those
%routers should have heard the same route from the router that learned
%the route via eBGP; if the iBGP topology is configured such that the
%routers in the top-level of the route reflector are not fully meshed,
%iBGP may not propagate routes to every router in the AS.
%(Incorrect iBGP configuration can also cause routing loops and
%oscillations; we discuss this in Section~\ref{sec:layer}.) 
\vspace{0.05in}

{\bf Each router is unaware of the state at other routers; this lack of
  information may result in incorrect or suboptimal routing.}.
  Implementing BGP's many features on the routers makes these features
  difficult to reason about.  For example, replication of functionality
  that is intended to
  improve reliability can cause forwarding loops, and a feature intended
  to prevent routing instability can slow convergence.  A routing
  architecture should implement these features in a 
  module that has
  a complete view of the network state, rather than in the routers
  (each of which only has a partial view of network state); doing so
  would allow that module to ensure sensible, consistent network-wide
  route assignment and override any feature interactions that cause
  incorrect routing. 
%  Controlling these protocol interactions centrally also makes it much
%  easier to physically replicate the entity performing route
%  computation.

%
% route-flap damping
\vspace{0.05in}
\noindent
{\em Problems:} 
% MED
%As another example, BGP's MED attribute allows an AS to express
%preference for how its neighbor selects an egress point for sending
%traffic (\ie, ``cold-potato'' routing).  Because MEDs are
%compared only between routes learned from the same AS, a router cannot
%form a total ordering of routes, which can cause persistent route
%oscillation~\cite{Griffin2002}.
% RR replicas
A router typically has iBGP sessions to multiple
route reflectors to improve reliability.  When a route reflector
fails, protocol oscillation and forwarding loops can arise if the
second route reflector has a different view of the best routes.
Placing the two route reflectors close to each other reduces these
kinds of inconsistencies but introduces fate sharing (\ie, the risk of
shared failures). 
%On the other hand, we believe that 
As another example, BGP route flap damping suppresses
unstable routes that change frequently~\cite{rfc2439}.  Unfortunately,
sometimes a 
single failure can trigger many advertisements that can mistakenly
activate route flap damping~\cite{Mao2002}.  Network operators must work
backwards to select configuration parameters that prevent erroneous
damping.

\vspace{0.05in}
\noindent
{\em Solution:} 
An entity that performs route
computation using a consistent view of available routes and network
topology can be replicated using standard distributed 
systems algorithms. Unlike route reflectors, each replica would assign
the same route to each router, independently of its location in the
network.
A module with knowledge of the routes assigned to every router in the
AS could also detect when route changes are caused by path exploration and
avoid unnecessarily suppressing a route.

\vspace{0.05in}
%Whenever possible, routing-protocol functionality should be decoupled
%from the routers.  That is there is no fundamental reason why a
%separate, vendor-independent device could not choose paths on behalf
%of the routers.  We believe that much of the research community's
%affection for overlay networks comes from the desire to control
%Internet routing and the frustration that the IP routing architecture
%remains largely closed to innovation.  Separating the path-selection
%process from the routers would enable exactly the kind of
%experimentation enabled by overlays today, but with the hope of
%directly influencing the behavior of the {\em underlay\/} network.

%%XXX Mention vendor-specific configuration? XXX

%Allowing each router in the AS to perform path selection introduces a
%A participant in a dynamic routing protocol updates its path selection
%as available routes change.
%If the routing decisions at each router are coupled with
%the propagation of the routing messages themselves, inconsistent (and
%incorrect) paths can result as updates propagate.  

\paragraph{Control Routing Protocol Interaction}\label{sec:layer}

Dividing functionality into distinct modules with clear interfaces can
control complexity.  In the routing system, the {\em IGP\/} computes
paths between routers in an AS, {\em eBGP\/} computes paths between ASes,
and {\em iBGP\/} propagates eBGP-learned routes throughout an
AS.  At a higher 
layer, {\em overlay networks\/} route traffic along one or more end-host
hops, abstracting the IP substrate entirely.
%In this subsection, we argue that the process of network protocol design
%and operation should respect this layering.
%
%As shown in Figure~\ref{fig:today}, an AS's border routers use eBGP to
%exchange routes for global destinations with other ASes; we call the
%layer at which eBGP routes are exchanged the {\em eBGP layer}.  Each
%eBGP-speaking router uses iBGP to exchange routes for global
%destinations with other routers in its AS.  We will refer to the layer
%at which iBGP routes are exchanged the {\em iBGP layer}.  If one of
%these routers selects a route for some destination that it learned from
%another router via iBGP, it will forward packets for that destination to
%that other router using the IGP path.  We will refer to the layer at
%which routers forward packets within an AS as the {\em IGP layer}.
%
Unfortunately, the modules in today's
interdomain routing system interact in the following undesirable ways:

{\bf Hard-wired interactions between eBGP and the IGP constrain an
operator's control over path selection.}  Although the internal
topology should have some influence on BGP routing decisions (\eg, 
it allows nearest-exit routing), a
router's choice of egress point should be relatively insensitive to
small IGP changes.
%, but it should still use IGP
%information when it informs BGP route selection.
%% Although the
%% routing architecture should {\em enable\/} IGP path costs to influence
%% the BGP routing decision, the interaction between the IGP and eBGP
%% layers {\em should not be hard-wired\/}.
%A router may learn multiple routes to a destination over several iBGP 
%sessions.  That router then applies a {\em de facto}
%``decision process'' to select a single best route to a destination.
%One step in that decision process dictates that routes with lower cost
%IGP paths be preferred over routes with longer IGP paths to an exit
%point.  
%The interplay between BGP and IGP limits operator control in two ways:
%(1)~minor changes in IGP routing can cause disproportionate effects in
%BGP routing, and (2)~to effect changes in intra-AS routing for network
%engineering, network operators are often forced to adjust the interdomain
%protocol configuration.
%
%Despite the benefits of hot-potato routing, 
%This happens because the BGP path selection process introduces
%dependencies on the IGP, a layering violation.

\vspace{0.05in}
\noindent
{\em Problem:} 
The BGP decision process uses the IGP path cost to break
the tie between two ``equally good'' routes. Internal events, such as
link failures, planned maintenance, or traffic engineering often lead to
changes in the IGP path costs.  These IGP changes can cause a router to
change its best {\em BGP\/} route, causing abrupt, unwanted traffic
shifts~\cite{teixeira2004b}.  Additionally, an operator may sometimes {\em
want\/} to redirect traffic from one egress link to another.  Today,
this requires complex manipulation of the BGP import policies to make
some egress points less attractive than others~\cite{Feamster2003e}.

\vspace{0.05in}
\noindent
{\em Solution:} 
With better control over the interactions between eBGP and IGP, an
operator could directly assign new routes to some routers without changing
BGP routing policies.
\vspace{0.05in}

{\bf Inconsistencies between iBGP and IGP can cause
forwarding loops and route oscillation.}  Operators can test that
their iBGP configuration satisfies sufficient conditions for
correctness~\cite{Griffin2002}, but this 
approach is not robust because operators commonly misconfigure
iBGP~\cite{Feamster2004h}.  The routing architecture should
explicitly {\em enforce\/} correctness constraints.  

\vspace{0.05in}
\noindent
{\em Problem:}
An
iBGP route reflector selects and distributes one best BGP route for each
destination prefix.  As a result, the route-reflector clients do not
necessarily make the same BGP routing decisions as they would in a
full-mesh iBGP configuration.  In particular, a route reflector and
its clients may have different IGP path costs to the
egress routers, leading to different BGP routing decisions, as shown 
previously in Figure~\ref{fig:rr}. 
%A route reflector and its clients may have a different view of the IGP
%path costs, which could result in inconsistent paths to the same
%egress point.
These inconsistencies can lead to protocol oscillations or persistent
forwarding loops~\cite{Basu2002,Griffin2002,rfc3345} if a router
forwards a packet toward one egress point via a router that has
selected a BGP route with a {\em different} egress point.  These
``deflections'' can also cause the AS-level forwarding path to differ
from the BGP AS path, which can complicate debugging~\cite{Mao2003}.

\vspace{0.05in}
\noindent
{\em Solution:} 
Rather than being agnostic about IGP forwarding paths, the routing
architecture could use the available knowledge to explicitly enforce
consistency in router-level forwarding paths.
\vspace{0.05in}

%decision process, can cause persistent route oscillation if route
%reflectors and their clients are not configured
%correctly~\cite{Griffin2002b,rfc3345}.  Griffin {\em
%et al.} observed that iBGP and IGP suffer from {\em asymmetry}, where
%the path over which data travels (determined by the IGP) is not
%necessarily the reverse of the control traffic
%(determined by the iBGP signaling graph)~\cite{Griffin2002}.  Asymmetry
%between iBGP and IGP can cause loops or ``deflections'' (\ie, instances
%where the forwarding path does not match the path indicated by the
%corresponding route), because routers along the IGP
%path may not agree on the same route to a
%destination~\cite{Dube99,Griffin2002}. 

%The interactions between iBGP and IGP complicate route selection and
%router-level forwarding paths. Even if an operator could predict the
%route that each router selects for a destination, it is still often
%difficult to determine whether the path over which packets are forwarded
%will actually follow that route.  For example, although conventional
%wisdom holds that iBGP route reflection simulates the effects of a full
%mesh iBGP topology, route reflection does not in fact do this.  A route
%reflector effectively executes the BGP decision process on behalf of its
%clients {\em under the assumption that its clients would have selected
%the same best route}.  Thus, the route that a router selects depends on
%the configuration of the route reflector hierarchy.  Configuring the
%route reflector hierarchy to avoid persistent forwarding loops requires
%extreme care.
%
%Although following proposed configuration guidelines can mitigate the
%undesirable interactions between iBGP and IGP, these work\-arounds
%avoid the fundamental protocol design problems that must be addressed as
%part of a long-term solution.  Interactions between iBGP and IGP become
%more considerably complex as the number of routers in an AS increases.
%One way of mitigating this complexity is to divide a large AS into
%multiple smaller ASes (or ``confederations'') with fewer routers.
%However, each configuration must still use iBGP to propagate routes
%internally, and the resulting larger network now comprises multiple ASes
%which are subject to misconfigurations themselves.  Additionally,
%previous work has derived sufficient conditions that must exist for iBGP
%to converge to a stable, loop-free routing that correctly propagates
%routes~\cite{Griffin2002}, but the protocol does not explicitly enforce
%these conditions.
%Although this work
%presents a practical, immediate way to avoid certain iBGP correctness
%problems, 
%the sufficient conditions on routing configuration are merely
%The intra-AS routing protocol should be designed to {\em
%enforce} these correctness constraints and explicitly prevent unwanted
%interactions between routing layers.\footnote{Tunneling (expand this)
%(\eg, via MPLS or IP-in-IP encapsulation) prevents persistent
%forwarding loops because forwarding decisions within the AS are
%determined by tunnels, rather than destination-based routing.
%Essentially, this type of tunneling enforces a layer separation between
%iBGP and the intra-AS routing. XXX Doesn't solve everything.}

{\bf Interactions between overlay networks and the underlying
network can degrade performance}.  
Overlay networks measure end-to-end path performance and tune routing
at the edge of the network, but they typically lack (1)~detailed {\em
measurements\/} of traffic and routing that would help them make
better decisions and (2)~direct {\em control\/} over IP-layer
protocols and mechanisms. The routing architecture should provide the
information and control that overlays need via a well-defined
interface. 
%

\vspace{0.05in}
\noindent
{\em Problem:} 
Route control products~\cite{www-routescience,www-sockeye}
help multihomed ISPs select upstream routes for each destination,
whereas end-host overlays such as RON~\cite{Andersen01} circumvent
failures and congestion by directing traffic through an intermediate
host.
% why are they harmful
Because they lack complete information about routing and
traffic-engineering optimizations, these overlays sometimes increase
congestion and decrease the effectiveness of traffic engineering in the
underlay network~\cite{Qiu2003}, which can degrade user performance.

\vspace{0.05in}
\noindent
{\em Solution:} 
With more direct control, overlays could operate more efficiently (\eg,
by not sending the same traffic over congested links at the network
edge~\cite{Jannotti2002}).  With more information about routing
dynamics, overlays could
pre-emptively avoid some outages~\cite{Feamster2003}. 

%% A control overlay, with a node in each AS, could monitor the behavior of
%% the routing system and drive the selection of paths in the under
%% network.  With an appropriate interface, this control overlay could
%% provide invaluable support to conventional overlay networks.


%Overlays improve end-to-end path performance and increase routing
%flexibility, but giving overlays the appropriate interface to the IP
%layer can improve network performance further.  Rather than creating
%control loop interactions between overlays and IP routing, the network
%architecture could provide the overlay with some level of control over
%the IP layer path to a destination.  Conversely, the IP layer could
%take ``hints'' from the overlay about end-to-end path performance and
%incorporate this into path selection at the IP layer.


% {\em BGP confederations or separate ASes:} problem still exists in
%      each confederation or mini-AS (though maybe RRs can be avoided), and
%      you now have to model a multi-AS network.  (I need to read ``on being
%      the right size'').




\paragraph{Support Flexible, Expressive Policies}

%% <<<<<<< strawman.tex
%% When used appropriately, abstraction can mitigate the complexity in
%% complex systems.  

%% However, the indirection that interdomain routing employs actually
%% exposes complexity rather than hiding it.  BGP's configuration and
%% operation involves multiple levels of indirection, but the indirection
%% it uses serves to complicate the protocol by {\em exposing mechanism},
%% rather than abstracting these details.  An interdomain routing protocol
%% should allow operators to specify tasks at the appropriate level of
%% abstraction, in a way that hides the mechanistic details of the routing
%% protocol rather than exposing them.  Note that these problems are not
%% merely configuration problems but involve fundamental protocol design
%% principles.
%% =======

%Much of the complexity of interdomain routing comes from the need to
%support competing domains with complex, sometimes conflicting, policy
%goals.  In fact, 
The interdomain routing architecture must support flexible, expressive
policy.  The need for greater flexibility in selecting and exporting
routes has driven many of the extensions to BGP over the past fifteen
years, and we believe this trend is likely to continue.  Although BGP is
highly configurable, its operation is controlled by {\em indirect\/}
mechanisms that expose details rather than abstracting them.
Architectural simplifications and better abstractions can simplify
configuration languages and make policy specification simpler and more
expressive.  The following points illustrate why today's routing
architecture does not satisfy these goals:

%Complex systems often exhibit emergent properties and unintended side
%effects. For example, a feature that is added for a certain reason may
%interact with others in strange and unexpected ways.  To control these
%types of effects, system designers often apply {\em modularity}:
%dividing the components of a system into groups of components such that
%each group can be reasoned about independently.  Designers aim to divide
%a system into modules in a way that minimizes the number of
%interconnections between modules.  Modules then interact with each other
%and with the environment via well-defined interfaces.  BGP's design
%gives rise to feature interactions, unnecessarily exacerbating
%complexity.  BGP's configuration and operation involves multiple
%unnecessary levels of indirection, which complicate the protocol by {\em
%exposing mechanism}, rather than abstracting these details.  In the
%remainder of this subsection, we describe how BGP's design restricts
%flexibility and gives rise to complicated feature interactions.

{\bf BGP's mechanisms preclude the expression of certain policies
and make others difficult to express.}  Network operators influence the
outcome of the BGP decision process by configuring policies that modify
the attributes of BGP routes.  Better configuration languages would be
helpful, but the architecture should also provide more flexible support
for assigning paths to routers.  

\vspace{0.05in}
\noindent
{\em Problem:} 
Moving
traffic from one inter-AS link to another requires~\cite{Feamster2003e}:
(1)~identifying the subset of prefixes that carries the desired amount of
traffic, (2)~determining how to express that subset (\eg, by a common
AS path regular expression), (3)~modifying the import policies on one or
more routers to assign a smaller ``local preference'' for routes
matching those expressions, and (4)~observing the resulting traffic flow
and iterating as necessary.  

\vspace{0.05in}
\noindent
{\em Solution:} 
Although ``what if'' tools can help predict
the effects of policy changes~\cite{Feamster2004}, the routing architecture
should allow an operator to move traffic by {\em explicitly} assigning paths.

%% Additionally, BGP has a ``hold timer'' setting that
%% specifies how the router will wait in between BGP ``keepalive'' messages
%% before deciding that a session has failed.  The hold timer setting is an
%% indirect mechanism for expressing how fast a router should detect a
%% failure.  The setting of the hold timer is somewhat of an art: for
%% example, a short hold timer interval allows for quicker detection of
%% failed sessions, but the interval must be longer than IGP convergence
%% delay to prevent spurious session resets.

{\bf BGP's mechanisms impede multiple ASes from cooperating in
selecting routes that satisfy their goals.}  ASes must cooperate to
ensure end-to-end reachability, but today's routing architecture does
not directly support this type of cooperation.  Interdomain routing
policies are a tussle space~\cite{clark2002}: an AS must balance the
dependence on its neighbors for good connectivity to the rest of the
Internet and competition with neighbors for customers and revenue.
Operators must currently resolve these conflicts outside of the
infrastructure, but
the architecture should directly support route
selection based on negotiated preferences or financial incentives.

\vspace{0.05in}
\noindent
{\em Problem:} 
Suppose one AS wants to advertise a backup route to its
neighbor.  These two ASes must first negotiate a backup
``signal'' out of band.  The AS advertising the route must then modify
its export policies to attach this signal to the backup route, and the
neighbor must modify the import policies on its routers to lower
the ``local preference'' value for routes with this community.

\vspace{0.05in}
\noindent
{\em Solution:} 
Because route negotiation is fundamental to inter-AS
cooperation, the interdomain routing should support it directly. 

%Additionally, an
%upstream neighbor might have a strong preference for the second route
%but would not have an easy way to convey this interest.  Although such
%preferences could be conveyed out of band~\cite{mahajan2005}, each AS
%would need to reconfigure each router appropriately.


%% Finally, {\em router configuration languages dictate low-level
%% specification of protocol behavior, rather than high-level policy or
%% operator intent}.  Operators must express constraints on route
%% distribution using indirect methods of expression.  To do so, they
%% typically implement filters that control the distribution of routes
%% based on whether a route's attributes matches specified pattern.  For
%% example, an operator may configure a router to filter routes that have a
%% certain ``community'' attribute~\cite{rfc2519}.  Controlling route
%% propagation using communities involves definitions at three levels of
%% indirection: (1)~mapping a BGP session to the names of import and export
%% policies, (2)~mapping the name of each policy to a set of clauses, each
%% of which defines conditions for accepting or rejecting a route and
%% modifying route attributes, and (3)~defining various variables that are
%% used in the clause of a policy statement (\eg, specifying what AS
%% regular expression number $x$ refers to).  These levels of indirection
%% make controlling route distribution complicated and error-prone because
%% they specify {\em how} the protocol should implement a certain behavior,
%% rather than directly specifying the policy to implement.  A
%% protocol and configuration language that supported direct expression of
%% high-level policy would ease the burden of network management.
%  communities+ACLs for aggregation~\cite{rfc2519}
%  and imposing relationships like customer, peer, and provider

%An interdomain routing protocol should allow operators to specify tasks
%at the appropriate level of abstraction, in a way that hides the
%mechanistic details of the routing protocol rather than exposing them.
%Note that these problems are not merely configuration problems but
%involve fundamental protocol design principles.  Better router
%configuration languages would mitigate some of these problems, but many
%of the problems associated with this interface result from incorrect
%{\em protocol} mechanisms.  Thus, allowing operators to configure
%routers at the appropriate level of abstraction involves architectural
%changes, as well as improvements to the configuration language itself.






