%\subsection{Research Agenda}
\label{sec:conclusion}

%We have argued that RCP simplifies interdomain routing and makes
%many configuration and management tasks easier, but we have left many
%details regarding how RCP will accomplish these tasks as future
%work.
%In other words, this paper did not present complete solutions to
%interdomain routing problems, but rather illustrated that RCP makes
%many problems tractable by reducing routing complexity. 
In addition to addressing the challenges discussed in
Section~\ref{sec:weaknesses}, we intend to design specific algorithms
and techniques for how RCP can improve interdomain routing in the
following areas:


{\bf Configuration languages.} RCP simplifies the underlying
routing mechanisms, which can in turn simplify configuration languages.
For example, configuring routing policy using RCP obviates the need for
implementing high-level tasks with communities and complex import and
export policies on individual routers.  We believe that locating
configuration state at the RCP should make it easier for operators to
specify high-level tasks, leaving the mechanistic details of {\em how}
these tasks are accomplished to RCP.  

{\bf Correctness and security.}  Correctness and security should be
intrinsic to the interdomain routing architecture.  RCP should impose
{\em invariants} on network configuration to guarantee correctness.  For
example, RCP can enforce consistent path assignment, as we described in
Section~\ref{sec:arch}.  RCP could also enforce other correctness
properties~\cite{Feamster2003b} by enforcing invariants.  Defining
what those invariants should be is an area for future work.

{\bf Troubleshooting and diagnostics.} Because RCP is effectively a
 repository of the routing state
for an AS, it can help operators debug routing and performance problems.
Of course, for RCP to be a useful tool for troubleshooting and
diagnostics, we must determine: (1)~the problems that network operators
commonly need to diagnose and (2)~the state that RCP must maintain
to be able to answer these questions.

{\bf Routing efficiency.}  We intend to explore how RCP could
improve routing efficiency.  For example, RCP could make routing
more efficient by aggregating prefixes for a particular router's
forwarding table if it could determine that the router would make the
same forwarding decision for all of the more specific routes.  An open
question is how RCP can efficiently determine when
aggregating contiguous prefixes is possible.  Additionally, because
RCP has a complete view of network state within an AS, we believe
that it could be used to selectively advertise more specific prefixes
for backup or inbound traffic engineering.



%% \begin{itemize}
%% \itemsep=-1pt
%% \item reliable, robust system (coping with failures)
%% \item  how to improve configuration language/specify behavior.  note
%%   that our protocol improvements also make this easier
%% \item  how to take advantage of the ability to connect route servers in
%%     different domains
%% \end{itemize}

%%  {\em Complete centralization of intra-AS routing:} Getting rid of the 
%%       IGP (where route server tells everyone shortest paths, too).  Need at 
%%       least some minimal functionality in the IGP so can reach the route 
%%       server -- use to segue into questions about exactly how much can 
%%       be done in a (logically) centralized fashion


%% We have presented three architectural principles for
%% interdomain routing (more complexity out of the core, separate routing
%% layers, and facilitate flexible and modular behavior), described why the
%% current architecture does not support these principles, and presented a
%% new routing architecture, the Routing Control Platform (RCP), that
%% cleanly separates the control and data paths.

%% When deployed within a single AS, the RCP significantly reduces routing
%% complexity by separating routing control from the routers that forward
%% packets.  This separation allows better control over the interactions
%% between routing layers and enables more flexible and expressive routing
%% policies; we have described how deploying the RCP, even within one AS,
%% can provide benefits such as improved convergence, better routing
%% stability, routing table compression, and more flexible path selection,
%% while requiring only simple changes to the {\em configurations} of
%% existing routers.  Because the RCP provides significant benefits
%% independently of whether other ASes deploy it, we envision that many
%% ASes might actually deploy RCPs.  If RCP were to see widespread
%% deployment, the innovation possibilities for interdomain routing are
%% great: the RCPs could operate interdomain routing protocol itself,
%% providing a rapid deployment path for innovative changes to interdomain
%% routing.
 
