%% \cs{T}he complexity of Internet routing configuration and protocol
%% dynamics mandates a framework for understanding and manipulating
%% Internet routing at a level of abstraction that facilitates reasoning
%% about whether or not the protocol is behaving ``correctly''.  Naturally,
%% the first step in this process is defining what it means for a routing
%% protocol to operate correctly in the first place.  Specifying
%% correctness is not easy, because it requires distilling a high-level
%% description of how the routing protocol should behave from the myriad
%% technical minutia of the operation of the protocol.

\cs{T}he flexibility offered by Internet routing configuration
allows the protocol to scale well and enables operators to express a wide
variety of policies, but it increases the complexity of the system.  This
complexity makes it difficult to reason about the behavior of Internet
routing, leading to {\em ad hoc} fixes to observed problems that
ultimately only worsen this complexity.  Today, network operators (who
continually tweak routing configuration) and protocol designers
(who repeatedly propose ``point'' solutions to various problems) have
no way of reasoning about whether their modifications to Internet
routing will
operate as intended.  Worse yet, operators and designers do not even
have a {\em specification} of properties that Internet routing should
satisfy.  This chapter seeks to remedy this situation by specifying
correctness properties
that any routing protocol should satisfy.

We introduce three properties to
classify the behavior of a routing protocol.  We briefly describe
these properties below and explain why they are critical for
correct routing.

\begin{enumerate}
\itemsep=-1pt
\item {\em Route validity} states that if a router has a route to a
destination, then a usable path corresponding to that route exists in
the underlying topology.  If route
validity is violated, then end users could experience a failure of
end-to-end connectivity, because routers could forward packets along
non-existent paths.
%; for example, routers may ``blackhole'' traffic,
%or the traffic may get caught in a loop and may never reach the
%destination.
\item {\em Path visibility} states that if there is a usable path
between two nodes, then the routing protocol will propagate information
about that path.  A failure of path visibility could disrupt
end-to-end communication by preventing two connected nodes from learning
routes between one another.
\item {\em Safety} states that the routing protocol converges to a
stable route assignment, regardless of the order in which routing
messages are exchanged.  A routing protocol that violates safety will
induce persistent route oscillations, causing routing changes that
are unrelated to changes in topology or policy.
\end{enumerate}
\noindent
This chapter defines and investigates these properties and
demonstrates how they can 
deepen our understanding of network routing.
This correctness specification addresses {\em static} properties of
network routing, {\em not} dynamic behavior (\ie, its response to
changing inputs, convergence time, etc.). Internet routing, like any
distributed protocol, may experience periods of transient incorrectness
in response to changing inputs, but we are concerned with persistent
misbehavior.

Chapter~\ref{chap:rcc} presents an approach to detect when two
of these properties (route validity and path visibility) are violated.
Chapter~\ref{chap:sandbox} exploits
these properties to predict which
of many possible routes each router in the network will select.
Chapter~\ref{chap:policy} deals with the 
challenges of guaranteeing safety, an inherently global property.
%Chapter~\ref{chap:beyond} discusses how both dynamic analysis and
%routing protocol improvements can help guarantee these properties.

%Previous work in wide-area protocol design
%has focused on specific modifications to BGP that fix a particular
%problem but often spur unintended negative side effects.  Furthermore,
%designers of new wide-area routing protocols require a mechanism that
%enables them to reason about the circumstances under which the protocol
%will behave ``correctly''.  To help reason about modifications to
%existing routing protocols and to aid in the sound design of new ones in
%the future, we propose that routing protocols be classified in terms of
%specific high-level properties.

After we introduce some basic terminology in
Section~\ref{sec:definitions}, we motivate and describe the correctness
properties.  Sections~\ref{sec:validity_def},
\ref{sec:visibility_def},~and
\ref{sec:safety_def} 
describe route validity, path visibility, and safety, respectively, and
explain how various aspects of the operation and configuration
of the Internet's routing protocols can cause each of these to be
violated in practice.   
