%%%
%%% conclusions.tex
%%%

\section{Summary}
\label{sec:sandbox_concl}

To perform everyday network engineering tasks effectively, efficiently,
and with minimal unnecessary changes to the live network, 
operators need a way to predict the behavior of a routing protocol
before deploying that configuration.  
%The model we have presented is a necessary step for advancing the state
%of the art of network engineering.
%, because it allows network operators (and protocol designers) to {\em
%reason} about BGP's behavior, rather than blindly tweaking the
%protocol with no understanding of the complex dynamics and protocol
%interactions.  
This chapter has presented route prediction algorithms that predict the
outcome of BGP route selection based on only a static snapshot of the
network state.

In addition to helping network operators accomplish traffic engineering
tasks, these algorithms provide useful insight into the subtleties of
network-wide BGP route selection and suggest several directions for
improvements to the Internet routing system.  For instance,
network-wide BGP route prediction could be combined with traffic
measurements to help network operators select BGP configuration changes
that achieve various traffic engineering goals.  In addition, the
emulator could be combined with higher-level mechanisms that spot
misconfiguration or check that other constraints
are satisfied~\cite{Feamster2004h}.

Although the diagram in Figure~\ref{fig:modules} shows only three
stages, we envision that network operators could incorporate other
phases.  For example, another phase could
combine the predicted forwarding paths with traffic data to predict
the load on each link in the network.  Using the model for traffic
engineering assumes that traffic volumes are relatively stable, and
that they remain stable in response to configuration changes.  In
previous work, we found that prefixes responsible for large amounts of
traffic have relatively stable traffic volumes over long
timescales~\cite{Feamster2003e}.  Operators could use the routing model to
test configuration changes on reasonably slow timescales that affect
prefixes with stable traffic volumes.  A network operator could also
combine measurements or estimates of the traffic arriving at each
ingress router for each destination prefix~\cite{Feldmann2001b} with the
link-level paths to predict the load on each link in the network.
Another phase might evaluate the optimality of the these link-level
paths in terms of propagation delay or link utilization and could
search for good configuration changes before applying them on a live
network.

Finally, we note that modeling BGP routing is more difficult than
it should be.  In the future, we hope that routing protocol designers
will consider predictability as a design goal; as we describe in
Section~\ref{sec:discussion}, some of these simplifications that aid
protocol modeling also fix problems with protocol {\em operation}.
Routing protocols that are easy to model and reason about will make
everyday network engineering tasks more tractable.


%% The prototype depends on many inputs including router 
%% configuration files, BGP table dumps, and BGP session information 
%% for every BGP-speaking router in the AS.  In reality, operators may not 
%% have access to all of these inputs, and some inputs may be incomplete 
%% or out-of-date.  Producing approximate results in the absence of  
%% complete information is a promising area for future work.

%In this section, we describe how
%operators can improvise in the face of incomplete or missing inputs.
%% The import policy application module relies on router configuration
%% files and a priori knowledge of what updates will be advertised via eBGP
%% from neighboring AS's.  Because the latter is impossible to know, we
%% approximate this using the externally learned routes in the BGP table
%% dumps (including alternate routes).  When some, but not all, routing
%% tables are available, it may be possible to approximate the missing
%% routing tables given knowledge of the other routing tables.  For
%% example, one simplifying assumption that may be valid for many networks
%% is to assume that all eBGP-speaking routers hear the same routing
%% advertisements.  \textbf{XXX how valid is this assumption?}
