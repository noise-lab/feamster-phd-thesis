\section{Take-away Lessons}\label{sec:rcc_lessons}


In recent years, much work has been done to understand BGP's
behavior, and much has been written about the wide range of problems it
has.  Some argue that BGP has outlived its purpose and should be
replaced; others argue that faults arise
because today's configuration languages are not
well-designed.  We believe that our evaluation of faults in
today's BGP configuration provides a better understanding of the
types of errors that appear in today's BGP configuration and the
problems in today's configuration languages.  
%
%Our evaluation of real-world BGP configuration from operational networks
%suggests several higher-level lessons about the nature of today's
%configuration process.  
We now briefly explore how our findings may help inform the design
of Internet routing systems in the future.

First, operational networks---even large, well-known, and well-managed
ones---have faults.  Even the most competent of operators find it
difficult to manage BGP configuration.  In particular, iBGP is
misconfigured often.  In fact, in the absence of a guideline such as
Theorem~\ref{thm:vis}, it is hard for a network operator to know what
properties the iBGP signaling graph should have.  Ideally, network
operators should be able to configure an AS without having to worry
about whether these types of constraints are satisfied in the first
place; in other words, a network operator should not be allowed to
express a configuration that violate properties such as path visibility
and route validity.  In Chapter~\ref{chap:concl}
(Section~\ref{sec:rcp}), we discuss a system for disseminating BGP
routes within an AS called the Routing Control Platform
(RCP)~\cite{caesar2004,feamster:fdna2004}, which could explicitly
enforce properties such as route validity and path visibility.

%
Second, we found that route filters are poorly maintained.  Routes that
should never be seen on the global Internet (\eg, routes for private
addresses) are rarely filtered, and the filters that are used are often
misconfigured and outdated.  We can make significant strides towards
fixing these types of problems simply by changing the default behavior
of router filters.  For example, because private address space (\ie, as
specified in RFC 3330~\cite{rfc3330}) should not be advertised on the
global Internet, routers could, by default, prevent routes for this
address space from leaking across AS boundaries.  Of course, network
operators who required routes for this address space to be advertised
across AS boundaries (\eg, in cases where a single administrative domain
comprises multiple ASes) could configure that behavior as an {\em
exception}, but the default behavior would reduce the likelihood of
erroneous route leaks.

%
Third, the majority of the configuration faults that \rcc detected
resulted from the fact that an AS's configuration is distributed across
its routers.  Maintaining network-wide policy consistency appears to be
difficult; invariably, in most ASes there are routers whose
configuration appears to contradict the AS's desired policy.  A routing
architecture or configuration management system that enabled an operator
to configure the network from a centralized location with a high-level
language would likely prevent many serious faults.

%
Finally, although operators use tools that automate some aspects of
configuration, these tools are not a panacea.  In fact, we found cases
where the incorrect use of these tools {\em caused} configuration
faults.  This observation suggests that static configuration
analysis will play an important role in the configuration
workflow, regardless of future improvements in configuration languages
or routing architectures.  As long as the routing protocol offers
flexible configuration, the potential for incorrect behavior exists.
Our work has demonstrated that detecting incorrect behavior {\em
proactively} using static configuration analysis is not only
surprisingly effective, but it is also necessary for detecting faulty
configurations before they introduce erroneous behavior on a live
network. 



\section{Summary}
\label{s:rcc_concl}


Despite the fact that BGP is almost 10 years old, operators continually make
the same mistakes as they did during BGP's infancy.   Our work takes a
step towards improving this state of 
affairs by making the following contributions:
%\vspace*{-0.05in}
\begin{itemize}
\itemsep=-1pt
%\item We define a high-level correctness specification for BGP and
%  map that specification to conditions that can be tested with static
%  analysis.
\item We use the correctness specification from Chapter~\ref{chap:rlogic}
  to design and implement 
  \rccns, a static analysis tool that detects faults by analyzing the
  BGP configuration across a single AS.  With \rccns, network operators
  can find many faults {\em before} deploying configurations in
  an operational network.  \rcc has been downloaded by over seventy network
  operators.
\item We use \rcc to explore the extent of real-world BGP
      misconfigurations.  We have analyzed
      real-world, deployed configurations from 17 different ASes and
      detected more than 1,000 BGP configuration faults that had
      previously gone undetected by operators.
\end{itemize}



%framework to the design and implementation of \rccns, a tool that uses
%static analysis to verify BGP configuration correctness.  Our tool has
%helped operators debug real-world BGP configuration.  Third, we
%present our findings on BGP configuration errors and anomalies from
%static analysis of real-world BGP configurations.  This paper is the
%first to explore BGP configuration errors using analysis of real-world
%BGP configuration files.  Finally, we suggest concrete ways to improve
%BGP's correctness, distinguishing problems that should be fixed with
%protocol modifications from those that should be fixed with a better
%configuration language.

In light of our findings, we suggest two ways to make interdomain
routing less prone to configuration faults.  First, protocol
improvements, particularly in intra-AS route dissemination, could avert
many BGP configuration faults.  The current approach to scaling iBGP
should be replaced.  Route reflection serves a single, relatively simple
purpose, but it is the source of many faults, many of which cannot be
checked with static analysis of BGP configuration
alone~\cite{Griffin2002}.  The protocol that disseminates BGP routes
within an AS should enforce path visibility and route validity; the
Routing Control Platform offers one possible solution.

Second, BGP should be configured with a centralized, higher-level
specification language.  Today's BGP configuration languages enable an
operator to specify router-level {\em mechanisms} that implement
high-level policy, but the distributed, low-level nature of the
configuration languages introduces complexity, obscurity, and
opportunities for misconfiguration rather than design flexibility or
expressiveness.  For example, \rcc detects many faults in implementation
of some high-level policies in low-level configuration; these faults
arise because there are many ways to implement the same high-level
policy, and the low-level configuration is unintuitive.  Ideally, a
network operator would never touch low-level mechanisms (\eg, the
community attribute) in the common case.  Rather than configuring
routers with a low-level language, an operator should configure the {\em
network} using a language that directly reflects high-level
policies.



%, and this specification could be ``compiled'' into
%configuration statements that implement the policies.
% (Such a specification would also be much easier to analyze
%with a tool like \rccns.)



%XXX FUTURE WORK: IGP/iBGP...

%% The hard stuff, like iBGP, isn't going to get better.  Same with things
%% like MED osc.  For these, we need protocol changes.  This is part of our
%% ongoing work (\eg, route servers to impose a total ordering, scrap iBGP
%% to avoid partitions, guarantee consistent export, etc.)

%% Same thing with safety.  It's fundamental to the protocol (Arrow's
%% impossibility thm), and protocol changes must happen to solve this
%% problem correctly.

%% For other stuff that people mess up, we need better interfaces.  These
%% interfaces should reflect that there are only a handful of tasks that
%% operators need to perform.
