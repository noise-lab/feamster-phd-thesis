The Internet is composed of thousands of autonomous,
competing networks that exchange reachability information using an
interdomain routing protocol.  Network operators must continually
reconfigure the routing protocols to realize various economic and
performance goals.  Unfortunately, there is no systematic way to predict
how the configuration will affect the behavior of the routing protocol
or to determine whether the routing protocol will operate correctly at
all.
This dissertation develops techniques to reason about
the dynamic behavior of Internet routing, based on {\em static}
analysis of the router configurations, before the protocol ever runs on
a live network.  

Interdomain routing offers each independent network tremendous
flexibility in configuring the routing protocols to accomplish various
economic and performance tasks.  Routing configurations are
complex, and writing them is similar to writing a distributed program;
the (unavoidable) consequence of configuration complexity is the
potential for incorrect and unpredictable behavior.  These mistakes and
unintended interactions lead to routing faults, which disrupt end-to-end
connectivity.  Network operators writing configurations make mistakes;
they may also specify policies that interact in unexpected ways with
policies in other networks.  To avoid disrupting network connectivity
and degrading performance, operators would benefit from being able to
determine the effects of 
configuration 
changes {\em before deploying them on a live network}; unfortunately,
the status quo provides them no opportunity to do so.  This dissertation
develops the techniques to achieve this goal of proactively ensuring
correct and predictable Internet routing.

The first challenge in guaranteeing correct and predictable behavior
from a routing protocol is defining a specification for correct behavior.
We identify three important aspects of correctness---path visibility,
route validity, and safety---and develop {\em proactive} techniques for
guaranteeing that these properties hold.  Path visibility states that
the protocol disseminates information about paths in the topology; route
validity says that this information actually corresponds to those paths;
safety says that the protocol ultimately converges to a stable outcome,
implying that routing updates actually correspond to topological changes.

Armed with this correctness specification, we tackle the second
challenge: analyzing routing protocol configurations that may
be distributed across hundreds of routers.  We develop
techniques to check whether a
routing protocol satisfies the correctness specification 
{\em within a single independently operated network}.  
We find that much of the specification can be checked with static
configuration analysis alone.  We present examples of real-world routing
faults and propose a systematic framework to classify, detect, correct,
and prevent them.  We describe the design and implementation of
\rcc (``router configuration checker''), a tool that uses static
configuration analysis to enable network operators to debug
configurations before deploying them in an operational network.  We have
used \rcc to detect faults in 17 different networks, including several
nationwide Internet service providers (ISPs).  To date, \rcc has been
downloaded by over seventy network operators.  

A critical aspect of guaranteeing correct and predictable Internet
routing is ensuring that the interactions of the configurations {\em
across multiple networks} do not violate the correctness specification.
Guaranteeing safety is challenging because each network sets its policies
independently, and these policies may conflict.  Using a formal model of
today's Internet routing protocol, we derive conditions to guarantee
that unintended policy interactions will never cause the routing
protocol to oscillate.

This dissertation also takes steps to make Internet routing more
predictable. We present algorithms that help network operators predict
how a set of distributed router configurations within a single network
will affect the flow of traffic through that network.  We describe a tool
based on these algorithms that exploits the unique
characteristics of routing data to reduce computational overhead.  Using
data from a large ISP, we show that this tool correctly computes BGP
routing decisions and has a running time that is acceptable for many
tasks, such as traffic engineering and capacity planning.



%% Because the behavior of the Internet routing system also depends on the
%% behavior of other networks, it must also exploit reactive techniques; we
%% briefly discuss how certain reactive techniques can complement the
%% proactive techniques we present in this dissertation.  Furthermore, many
%% of the problems with correctness and predictability of the current
%% Internet routing infrastructure stem from the fact that today's
%% architecture was not designed with these goals in mind.  Carefully
%% considered changes to the Internet's routing protocol and architecture
%% could make it more intrinsically robust against incorrect and
%% unpredictable behavior.  We discuss several fundamental limitations of
%% today's interdomain routing architecture and propose new architectural
%% directions.
