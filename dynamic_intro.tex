\cs{T}o this point, we have focused on how static configuration analysis
can be applied to proactively achieve correctness and predictability in
Internet routing {\em within a single AS}, for which all router
configurations are available.  However, in some cases, a network
operator may wish to detect whether some {\em neighboring} AS has a
faulty configuration or is otherwise sending ``incorrect'' routing
information.  In Section~\ref{sec:dynamic}, we explore how dynamic
analysis can often be useful in helping operators diagnose whether a
neighboring AS is violating the route validity problem discussed in
Section~\ref{sec:validity} by not advertising equally good routes at all
interconnection points, which is a violation of common settlement-free
peering contracts~\cite{www-aol}.  An important theme to note in this
work is the complementary role played by static configuration analysis:
determining whether the contract violations are caused by a neighboring
AS or the configuration in the local AS also requires analyzing the
local configurations with \rccns.  

Static configuration analysis is useful for detecting faults in existing
configurations, but a longer-term goal should be to make routing
configuration more intrinsically correct and predictable in the first
place.  That is, a network operator should be actively discouraged (and
perhaps even prevented) from specifying configurations that induce
incorrect behavior.  Chapter~\ref{chap:policy} addressed this problem to
some degree by exploring how rankings might be restricted to explicitly
guarantee safety.  In Section~\ref{sec:rcp}, we explore how a different
{\em architecture} for Internet routing could make routing failures less
likely.  Specifically, we advocate moving responsibility for selecting
Internet routes to a separate system called the Routing Control Platform
(RCP), which actively monitors intra-AS topology and available routes
and computes the route that each router should select for each
destination.  RCP presents many systems-level challenges (\eg, can such
a system compute routes for a large AS fast enough without being
overloaded?); other recent work has built a prototype to address some of
these concerns~\cite{caesar2004}.  In this dissertation, we focus on how
RCP could help improve correctness and predictability.
